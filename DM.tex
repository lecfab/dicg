\documentclass[a4paper, 11pt]{article}

\voffset -0cm
\hoffset 0.0cm
\textheight 23cm
\textwidth 16cm
\topmargin 0.0cm
\oddsidemargin 0.0cm
\evensidemargin 0.0cm

\usepackage[ruled,vlined,linesnumbered]{algorithm2e}   % authors: last version of algorithm display

\usepackage{epsfig}
\usepackage{setspace}
\usepackage{fancyhdr}
\usepackage{amsmath}
\usepackage{amssymb}
\usepackage{graphicx}
\usepackage{url}
\newtheorem{Lemma}{Lemma}

\title{}
\author{}
\date{}

\newtheorem{qu}{Question}

\begin{document}

\begin{center}
	\LARGE \textbf{Assignment: ``Images et g\'eom\'etrie discr\`ete''\\2015}
\end{center}

\section*{Introduction}

The objective of this project is to design a binary shape
indexing/retrieval system.

We expect from you:
\begin{itemize}
\item A short report with a complete description of the your
  implementations choices (see below)
\item A C++ project (\texttt{CMakeLists.txt}  or \texttt{Makefile} plus couple of
  \textbf{commented} cpp program files or scripts).
\end{itemize}


\section*{Description}

Let us consider that we have solved the segmentation problem and that
we have a database of binary shapes. The challenge is to design shape
indexing and shape retrieval processes for such database. More
precisely, we consider a database of 70 classes with approx. 20 shapes
in each class. For instance, there are 20 different hammers in the
\texttt{hammer} class.


 The problem consists in the following process:
\begin{itemize}
\item Shape indexing: for each shape, we create a signature vector
  $\vec{v}\in\mathbb{R}^d$ of $d$ scalar quantities describing the
  shape. The signature should characterize the shape geometry and
  could distinguish classes: signatures of shapes in the same class
  should be \emph{close} and signatures of two shapes in
  different classes should be \emph{different}. 
  
  \item Shape retrieval: given a input shape, we compute its feature
    vector $\vec{w}$ and we output for instance the $k$ \emph{nearest}
    shapes for a given similarity metric, or \emph{distance}, on
    signature vectors. We could also output the class to which the
    input shape should belong to (classification problem). We could
    also return a probability measure for a shape to belong in a given
    class.
\end{itemize} 

As you may have seen, requirements and specifications are
imprecise. This reflects the fact that many solutions exist and we
would like you to explore your own. Hence, the shape descriptor could
be based either on contour information (length, curvature,...), or on
volumetric information (geometrical moments,...). Similarly, the
\emph{distance} between shape signature could be a simple $l_2$ norm
of the vectors or even better statistical similarity.

Whatever the choices you consider to design your indexing/retrieval
system, we would like you to have a complete theoretical and
experimental evaluation of your descriptor/system. For example, you
must clearly discuss about the following points:
\begin{itemize}
\item Can I demonstrate that my descriptor is  invariant  under translation, scaling,
  rotation, shear, partial occlusion... ?
\item Can I demonstrate (or experimentally check) the robustness of
  the descriptor in presence of noise ?
\item How can I evaluate the retrieval system (accuracy of the
  classification, precision, recall...)~?
  \item \ldots
\end{itemize}

\section*{Evaluation}

As you may have noticed, many solutions exist. To evaluate your work,
the following points will be considered:
\begin{itemize}
\item The relevance of the discussion on the proposed descriptor;
\item The quality of the experimental evaluation;
\item The quality of the  pros/cons discussion.
\end{itemize}
Technically, we will also evaluate:
\begin{itemize}
\item The quality of the code;
\item The fact that we can reproduce the results (e.g. with shell scripts);
\item Finally, the quality of the indexing/retrieval system.
\end{itemize}



\section*{Details}

The image database is available in the course github project (Part of the
MPEG-7 CE-Shape-1 Dataset, \texttt{assignment/ShapeIndexing/database}).


In \texttt{DGtalSkel}, you will find an example
(\texttt{PGMReader.cpp}) to load a PGM file with DGtal tools and to
convert pixels with intensity greater than 0 into a digital set.
\texttt{imgAddNoise} tool can be used to generate a noisy version of a
binary image using a Kanungo noise model (pixel value swapping
procedure from a power law on the distance to the boundary, see DGtal documentation).
\vspace{1cm}

\noindent The final projects archive file (named after the \emph{binome}
authors) must contain
\begin{itemize}
\item A report (PDF, 5-10 pages)
\item Source files with either a \texttt{makefile} or a
  \texttt{CMakeLists.txt}
\item Scripts/executable to evaluate the system (for example, taking
  an input image file and returning its class or the closest images).
\end{itemize}


\end{document}
