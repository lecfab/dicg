\documentclass{article}

\usepackage[latin1]{inputenc}
\usepackage[T1]{fontenc}
\usepackage[francais]{babel}
\usepackage{layout}
\usepackage[top=2cm, bottom=2cm, left=2cm, right=2cm]{geometry}
\usepackage{amsmath} % math alignement
\usepackage{amssymb} % symboles de maths avances
\usepackage{mathrsfs} % police calligraphique \mathscr (majuscules)
\usepackage{pgfplots} % traceur de courbes
\usepackage{stmaryrd} % doubles braquets
\usepackage{bbm} % apparemment il ne sert a rien
\usepackage{multicol} % plusieurs colonnes
\usepackage{dsfont} % symbole indicatrice \mathds{1}

\newcommand{\br}[2][1]{\llbracket #1,#2 \rrbracket}
\newcommand{\parr}[1]{\left(#1\right)}
\newcommand{\acc}[1]{\left\lbrace#1\right\rbrace}
\newcommand{\deq}{\overset{\Delta}{=}}
\newcommand{\transpose}{^\intercal}
\newcommand{\bin}[1]{\acc{0,1}^{#1}}


\title{Digital Images Project}
\author{Fabrice \bsc{L�cuyer} and Etienne \bsc{Desbois}}
\date{}

\newcommand{\Ac}{\mathcal{A}}
\newcommand{\Cc}{\mathcal{C}}
\newcommand{\Lc}{\mathcal{L}}
\newcommand{\Mc}{\mathcal{M}}
\newcommand{\Nc}{\mathcal{N}}
\newcommand{\Sc}{\mathcal{S}}
\newcommand{\Oc}{\mathcal{O}}
\newcommand{\Uc}{\mathcal{U}}
\newcommand{\Ic}{\mathcal{I}}

\newcommand{\punif}[2][n]{\mathbb{P}_{#2\leftarrow \Uc\parr{\bin{#1}}}}

\begin{document}

\maketitle

\section{Features}
\subsection{Image moments}
The first kind of features we use is called image moments. It is a purely mathematical way to describe the distribution of pixels in an image. The $p,q$ moment of a 2D set of pixels $\Sc$ is defined by:
$$m_{p,q}=\sum_{(x,y)\in\Sc}x^py^q$$

Low moments have intuitive explanations: $m_{0,0}$ is the number of pixels, $\hat{x}=\frac{m_{1,0}}{m_{0,0}}$ is the mean value of abscisse, $\hat{y}=\frac{m_{0,1}}{m_{0,0}}$ is the mean value of ordinate.

To obtain translation invariance, we take a new origin $(\hat{x},\hat{y})$ and define central moments:
$$\mu_{p,q}=\sum_{(x,y)\in\Sc}(x-\hat{x})^p(y-\hat{y})^q$$

Now we would like to have features with invariance by rotation. Hu's paper gives seven of them (page 7). Let us prove that the first one is indeed invariant:

First, we suppose the image is centered, which means $\hat{x}=\hat{y}=0$.
$$
\begin{aligned}
\phi_1=\mu_{2,0}+\mu_{0,2}&=\sum_{(x,y)\in\Sc}(x-\hat{x})^2(y-\hat{y})^0+(x-\hat{x})^0(y-\hat{y})^2
=\sum_{(x,y)\in\Sc}x^2+y^2
\end{aligned}
$$
Now define $(x'=x\cos\theta+y\sin\theta,y'=-x\sin\theta+y\cos\theta)$ obtained from $(x,y)$ by rotation of angle $\theta$.
$$
\begin{aligned}
\phi'_1&=\sum_{(x,y)\in\Sc}x'^2+y'^2
=\sum_{(x,y)\in\Sc}(x\cos\theta+y\sin\theta)^2+(-x\sin\theta+y\cos\theta)^2\\
&=\sum_{(x,y)\in\Sc}(x^2\cos^2\theta+y^2\sin^2\theta+2xy\cos\theta\sin\theta)+(x^2\sin^2\theta+y^2\cos^2\theta-2xy\cos\theta\sin\theta)\\
&=\sum_{(x,y)\in\Sc} (x^2+y^2)\underbrace{(\sin^2\theta+\cos^2\theta)}_{=1}\\
&=\phi_1
\end{aligned}
$$

The last step is scale invariance. We define the standardardized moments by
$\eta_{p,q}=\frac{\mu_{p,q}}{\mu_{0,0}^{\parr{1+\frac{i+j}{2}}}}$.

Suppose we have $(x'=\alpha x, y'=\alpha y)$ obtained frome $(x,y)$ by $\alpha>0$ scaling.

$$
\begin{aligned}
\Phi'_1=\eta'_{2,0}+\eta'_{0,2}&=\frac{\mu'_{2,0}+\mu'_{0,2}}{\mu_{0,0}'^2}
=\frac{\alpha^2(\mu_{2,0}+\mu_{0,2})}{(\alpha\mu_{0,0})^2}
=\frac{\mu_{2,0}+\mu_{0,2}}{\mu_{0,0}^2}=\Phi_1
\end{aligned}
$$

Similarly, we get 7 features $\Phi_1,\dots,\Phi_7$ as given in Hu, that are invariant under translation, rotation and scale.

\subsection{tes trucs Etienne}

\section{Classification}
\subsection{Method}
kNN ? neural nets ?
\subsection{Distance}

\section{Tests}
\subsection{Classification}
\subsection{Distanciation}

\end{document}
